\documentclass[12pt]{article}

\usepackage{sbc-template}
\usepackage{graphicx,url}
\usepackage{algorithm}
\usepackage{algpseudocode} 
\usepackage[utf8]{inputenc}  

     
\sloppy

\title{Relatório da Implementação de Projeto e Análise de Algoritmos}

\author{Camila Lopes\inst{1}, Victor Colen\inst{1} }


\address{Instituto de CIências Exatas e Informática \\ Pontifícia Universidade Catórila de Minas Gerais (PUC-MG)\\
  30535-901 -- Coração Eucarístico -- Belo Horizonte – MG
  \email{\{lopes.camila, victor.costa\}@sga.pucminas.br}
}

\begin{document} 

\maketitle

\section{Introdução}

Este relatório explora e implementa duas classes para a manipulação de grafos direcionados, e outra para grafos não-direcionados. O grafo direcionado foi implementado a partir de  uma lista de adjacência, e assim  realizada uma busca em profundidade, rotulando a aresta "de árvore", "de retorno", "de avanço" e "de cruzamento". Já o grafo não-direcionado, foi implementado a partir da matriz de adjacência e, desse modo, realizado uma busca em largura encontrando a excentricidade de um determinado vértice.

\section{Implementação}

\subsection{Grafo Direcionado}

A lista de adjacência é uma estrutura de dados eficiente em termos de memória, especialemente para representar grafos esparsos, permitindo uma fácil representação e iteração sobre os vértices conectados.

\subsubsection{Estrutura de Dados}
No grafo direcionado é necessário observar que para cada vértice na lista de adjacência é associado a uma lista de vértices adjacentes.

\begin{algorithm}
    \caption{Estrutura de Dados}
    \label{alg:estruturaDeDados_GD}
    \begin{algorithmic}
        \State $numVertices =$ O número de vértices do grafo 
        \State $adjacencyList =$ Lista de adjacência
    \end{algorithmic}
\end{algorithm}

\subsubsection{Métodos Principais}

Foram classificados dois métodos importantes, sendo eles o método de \textit{Adição de Aresta} e o método de \textit{Busca em Profundidade}.

\begin{algorithm}
    \caption{Adicionar Aresta}
    \label{alg:adicionarAresta_GD}
    \begin{algorithmic}
        \Require Um vértice de origem $sourceVertex$ e de destino $destinationVertex$
        \State Adiciona $destinationVertex$ à lista de adjacência de $sourceVertex$
    \end{algorithmic}
\end{algorithm}

\begin{algorithm}
    \caption{Busca em Profundidade (DFS)}
    \label{alg:DFS}
    \begin{algorithmic}
        \Require Um grafo com $numVertices$ vértices
        
        \State Inicializa um vetor $visited$ de tamanho $numVertices$ com \textbf{falso}, indicando que nenhum vértice foi visitado
        \State Inicializa um vetor $discoveryTime$ de tamanho $numVertices$ com $-1$, indicando o tempo de descoberta dos vértices
        \State Inicializa um vetor $lowTime$ de tamanho $numVertices$ com $-1$, indicando o menor tempo acessível de cada vértice
        \State Inicializa um vetor $parent$ de tamanho $numVertices$ com $-1$, indicando o pai de cada vértice na árvore DFS
        \State Define uma variável $time$ como $0$ para rastrear o tempo de descoberta

        \For{cada $vertex$ de $0$ a $numVertices-1$}
            \If{$vertex$ não foi visitado}
                \State Classificar a aresta como "Árvore".
                \State Continuar a DFS recursivamente.
            \Else
                \State Classificar a aresta como "Retorno", "Avanço" ou "Cruzamento" com base nas condições de tempo de descoberta.
            \EndIf
        \EndFor
    \end{algorithmic}
\end{algorithm}


\subsection{Grafo Não-Direcionado}

A matriz de adjacência é uma estrutura de dados útil para a verificação rápida de conexões entre quaisquer dois vértices, se destancando pela sua simplicidade de acesso e modificação, ainda que apresente um custo maior em termos de armazenamento, especialmente em grafos de alta densidade.

\subsubsection{Estrutura de Dados}
No grafo não direcionado temos a matriz da adjacência, a qual cada célula indica a presença ou ausência de uma aresta entre dois vértices.

\begin{algorithm}
    \caption{Estrutura de Dados}
    \label{alg:estruturaDeDados_GND}
    \begin{algorithmic}
        \State $numVertices =$ O número de vértices do grafo 
        \State $adjacencyMatrix =$ Matriz de Adjacência
    \end{algorithmic}
\end{algorithm}

\newpage

\subsubsection{Métodos Principais}
Foram classificados dois métodos importantes, sendo eles o método de \textit{Adição de Aresta} e o método de \textit{Busca em Largura e Cálculo de Excentricidade}.

\begin{algorithm}
    \caption{Adicionar Aresta}
    \label{alg:adicionarAresta_GND}
    \begin{algorithmic}
        \Require Dois vértices $vertex1$ e $vertex2$
        \State Defina $adjacencyMatrix[vertex1][vertex2] \gets 1$
        \State Defina $adjacencyMatrix[vertex2][vertex1] \gets 1$
    \end{algorithmic}
\end{algorithm}

\begin{algorithm}
    \caption{Busca em Largura (BFS) e Cálculo da Excentricidade}
    \label{alg:BFS_calculoDaExcentricidade}
    \begin{algorithmic}
        \Require Um vértice inicial $startVertex$
        
        \State Inicializa o vetor $distance$ de tamanho $numVertices$, com os valores iguais a $+\infty$
        \State Inicialize uma fila $bfsQueue$
        
        \State Defina o vértice inicial $distance[startVertex] \gets 0$
        \State Adicione $startVertex$ à fila $bfsQueue$
        
        \While{$bfsQueue$ não estiver vazia}
            \State Retira o vértice da frente da fila e explora seus vizinhos
            
            \For{$i = 0$ \textbf{até} $numVertices - 1$}
                \If{$adjacencyMatrix[currentVertex][i]$ \textbf{e} $distance[i] = +\infty$}
                    \State tualiza a distância e adiciona o vizinho à fila
                \EndIf
            \EndFor
        \EndWhile
        
        \State Inicialize $excentricity \gets 0$
        \For{$i = 0$ \textbf{até} $numVertices - 1$}
            \If{$distance[i] \neq +\infty$ \textbf{e} $distance[i] > excentricity$}
                \State Defina $excentricity \gets distance[i]$
            \EndIf
        \EndFor
        
        \State \Return $excentricity$
    \end{algorithmic}
\end{algorithm}

\subsection{Criação de Grafos Aleatórios}

Para auxiliar na análise das classes criadas foram implementadas funções para gerar grafos direcionados e não-direcionados de tamanho aleatório. Dessa forma, essas funções permitem que novos grafos sejam criados para análise a cada execução do código.

\newpage

\subsubsection{Pseudocódigo para Criar Grafos Direcionados Aleatórios}

\begin{algorithm}
    \caption{Grafo Direcionado Aleatório}
    \label{alg:createRandomDirectedGraph}
    \begin{algorithmic}
        \Require Um número aleatório de vértices $maxVertices$
        \State Gera um número de vértices $numVertices$ aleatório entre $1$ e $maxVertices$
        \State Cria um grafo direcionado com $numVertices$ vértices
        \State Gera um número $numEdges$ entre $0$ e $numVertices \times (numVertices - 1)$
        \For{$i = 0$ to $numEdges-1$}
            \State Selecione um vértice $source$ aleatório entre $0$ e $numVertices-1$
            \State Selecione um vértice $destination$ aleatório entre $0$ e $numVertices-1$
            \If{$source \neq destination$}
                \State Adicionar uma aresta do vértice $source$ para o vértice $destination$
            \EndIf
        \EndFor
    \end{algorithmic}
\end{algorithm}


\subsubsection{Pseudocódigo para Criar Grafos Não-Direcionados Aleatórios}

\begin{algorithm}
    \caption{Grafo Não-Direcionado Aleatório}
    \label{alg:createRandomUndirectedGraph}
    \begin{algorithmic}
        \Require Um número aleatório de vértices $maxVertices$
        \State Gera um número de vértices $numVertices$ aleatório entre $1$ e $maxVertices$
        \State Cria um grafo direcionado com $numVertices$ vértices
        \State Gera um número $numEdges$ entre $0$ e $(numVertices \times (numVertices - 1)) \div 2$
        \For{$i = 0$ to $numEdges-1$}
            \State Selecione um vértice $source$ aleatório entre $0$ e $numVertices-1$
            \State Selecione um vértice $destination$ aleatório entre $0$ e $numVertices-1$
            \If{$source \neq destination$ e $\neg$ $(graph.hasEdge(source, destination))$}
                \State Adicionar uma aresta do vértice $source$ para o vértice $destination$
            \EndIf
        \EndFor
    \end{algorithmic}
\end{algorithm}

\section{Conclusão}

Neste trabalho, foi explorado a manipulação de grafos direcionados e grafos-não direcionados e implementado os requisitos propostos, fornecendo uma estrutura eficiente para manipulação de grafos e permitindo análises variadas a cada execução por meio da criação de grafos aleatórios. Ademais, a modularidade das classes facilita tanto a manutenção quanto a expansão futura do código se exigido novos requisitos.



%\bibliographystyle{sbc}
%\bibliography{sbc-template}

\end{document}
